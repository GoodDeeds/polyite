\documentclass{article}


\begin{document}

\section{General}
\begin{description}
\item[logToFile] Specifies, whether to log to a file (true/ false)
\item[logFile] log file. \textsc{Polyite} versions the log files.
\item[islComputeout] Maximum number of computation steps, that some expensive
ISL operations may take. Set to 0 to disable. An operation is canceled, if the
limit is exceeded (e.g. the generating of a schedule is discontinued.).
\item[barvinokBinary] binary that permits access to \textsc{libbarvinok}.
\item[barvinokLibraryPath] Barvinok library path
\item[paramValMappings] Value mappings for context parameters (n=42,m=43)
\end{description}

\section{SCoP}

\begin{description}
  \item[benchmarkName] Name of the benchmark. Gets passed to the benchmarking script.
  \item[functionName] Name of the kernel function. Gets passed to the benchmarking script.
  \item[scopRegionStart, scopRegionEnd] Together form the name of the \texttt{LLVM} IR
    code region to be optimized. Gets passed to the benchmarking script.
  \item[irFilesLocation] Tells the benchmarking script where to find the source code (or \texttt{LLVM} IR).
  \item[seed] Random seed for \textsc{Polyite} (Not reasonable, if $\texttt{numScheduleGenThreads} > 1$). May be either an integer or NONE.
\end{description}

\section{Sampling}

\begin{description}
    \item[numScheds] total number of schedules to sample.
    \item[numScheduleGenThreads] The number of threads that generate schedules in parallel.
    \item[rayCoeffsRange] a value $b$ with $[1, b]$ being the value range for ray coefficients
        in schedule coefficient vector sampling.
    \item[lineCoeffsRange] a value $b$ with $[-b, b] \setminus {0}$ being the value range for line coefficients
        in schedule coefficient vector sampling.
    \item[maxNumRays] Maximum number of rays in the linear combination that forms one schedule coefficient vector
      must be one of
      \begin{itemize}
      \item $n$: use at most $n$ rays
      \item ALL: Use arbitrarily many rays
      \item RAND: When starting to sample a schedule matrix, randomly choose an
        upper bound for the number of rays.
      \end{itemize}
    \item[maxNumLines] Maximum number of lines in the linear combination that forms one schedule coefficient vector
      must be one of
      \begin{itemize}
      \item $n$: use at most $n$ lines
      \item ALL: Use arbitrarily many lines
      \item RAND: When starting to sample a schedule matrix, randomly choose an
        upper bound for the number of lines.
      \end{itemize}
    \item[probabilityToCarryDep] During sampling of search space regions, the probability
      that all schedule coefficient vectors in a dimension polyhedron $P_d$ carry at
      least one dependence.
    \item[maxNumSchedsAtOnce] Maximum number of schedules to sample from the currently
    selected search space regions
    \item[moveVertices] When modeling a search space region, move vertices with
    rational coordinates to neighboring vertices with integer coordinates that
    are inside the polyhedron.
    \item[rayPruningThreshold] Threshold to prune rays and lines that coordinates
    exceeding rayPruningThreshold (rational number (x/y) or NONE)
    \item[randSchedsTimeout] Timeout in seconds for generating one random schedule.
    \item[genSchedsMaxAllowedConseqFailures] Sampling of schedules fails after
    so many consecutive failures to generate another schedule.
    \item[completeSchedules] If set to true, \textsc{Polyite} makes sure that a schedule
    explicitly encodes every loop from the generated code.
    \item[linIndepVectsDoNotFixDims] Tweak to speed up schedule completion (true/ false)
  \end{description}

    \section{Schedule Tree Transformation and Simplification}
    \begin{description}
      \item[simplifySchedTrees] Enables schedule tree simplification (true/ false)
      \item[splitLoopBodies] Enables canonical representation of loop fusion (true/ false)
      \item[insertSetNodes] Insert set nodes where they are known to be legal (ISL is not ready for this.)

      \item the following options enable the different schedule simplification steps (true/ false)
      \begin{itemize}
      \item \textbf{schedTreeSimplRebuildDimScheds}
      \item \textbf{schedTreeSimplRemoveCommonOffset}
      \item \textbf{schedTreeSimplDivideCoeffsByGCD}
      \item \textbf{schedTreeSimplElimSuperfluousSubTrees}
      \item \textbf{schedTreeSimplElimSuperfluousDimNodes}
    \end{itemize}
      \item[numSchedTreeSimplDurationMeasurements] Number of measurements of the
      duration of schedule tree simplification. (may be NONE)
    \end{description}

    \section{Genetic Algorithm}
    \begin{description}
      \item[\textbf{probabilityToMutateSchedRow}] Actually the share of schedule
      dimensions that is altered by a mutation. (value between 0 and 1).
      \item[\textbf{probabilityToMutateGeneratorCoeff}] Actually the share of
      generator coefficients that are mutated by generator coefficient replacement.

      \item[\textbf{generatorCoeffMaxDenominator}] Maximum denominator of rational
      generator coefficient that are introduced by generator coefficient replacement.
      \item[\textbf{currentGeneration}] Index of the generation, from which the
      genetic algorithm should start. If the value is not zero, the presence of
      a file \texttt{<populationFilePrefix><index>.json} is expected, from which
      the schedules of the current generation can be loaded.

      \item[\textbf{maxGenerationToReach}] The index of the maximum generation to reach.
      \item[\textbf{regularPopulationSize}] Size of a single population.
      \item[\textbf{fractionOfSchedules2Keep}] Fraction of schedules that survive
      a generation and can reproduce. Format: (n/d)
      \item[\textbf{maxNumNewSchedsFromCrossover}] More than one schedule can
      result from one geometric crossover. This parameter controls the maximum number.
      \item[\textbf{optimizeParallelExecTime}] Controls whether to optimize
      sequential or parallel execution time.
      \item[\textbf{evolutionTimeout}] A timeout in seconds for mutations and
      crossovers. This is necessary, since many operations on polyhedra have at
      least exponential execution time in the worst case.
      \item[\textbf{shareOfRandSchedsInPopulation}] Share of randomly generated
      schedules in each generation. Set this to a small rational number. Format: (n/d)

      \item[\textbf{replaceDimsEnabled}] Enable mutation by dimension replacement (true/false)
      \item[\textbf{replacePrefixEnabled}] Enable mutation by schedule prefix replacement (true/false)
      \item[\textbf{replaceSuffixEnabled}] Enable mutation by schedule suffix replacement (true/false)
      \item[\textbf{mutateGeneratorCoeffsEnabled}] Enable mutation by generator coefficient replacement (true/false)
      \item[\textbf{geometricCrossoverEnabled}] Enable geometric crossover (true/false)
      \item[\textbf{rowCrossoverEnabled}] Enable row crossover (true/false)
      \item[\textbf{initPopulationNumRays}] Works like \textbf{maxNumRays} but for the generation of the initial population.
      \item[\textbf{initPopulationNumLines}] Works like \textbf{maxNumLines} but for the generation of the initial population.
      \item[\textbf{useConvexAnnealingFunction}] Switch between convex and concave annealing function. Due to a bug, convex is concave (true/false).
    \end{description}

    \section{Random exploration with LeTSeE's Search Space Construction}
    \begin{description}
      \item[\textbf{boundSchedCoeffs}] Enable or disable the limitation of the
      schedule coefficients to $\lbrace -1, 0, 1 \rbrace$. (true/false)
      \item[\textbf{completeSchedules}] Allows to choose whether to append
      dimensions to the sampled coefficient matrices to encode all loops
      explicitly, or not. (true/false)
    \end{description}

    \section{Evaluation}
    \begin{description}
      \item[evaluateScheds] Enables schedule evaluation.
      \item[numMeasurementThreads] Number of benchmarking threads. Equals the
        number of simultaneously queued \textsc{SLURM} jobs.
      \item[measurementCommand] command to run the measurement script. Must run
        synchronously and permit communication via STDIN/ STDOUT.
      \item[numactlConf] \textsc{NUMACTL} config that is passed to the measurement command.
      \item[evaluationSigIntExitCode] The measurement should terminate with this
      exit code, when interrupted or terminated from outside.
      \item[measurementTimeout] Timeout for schedule evaluation in seconds.
      \item[compilationTimeout] Timeout for compilation in seconds. Should be $<=$ measurementTimeout (can be set to NONE)
      \item[measurementWorkingDir] Working of the measurement command.
      \item[measurementTmpDirBase] base directory for temporary working directories.
      \item[measurementTmpDirNamePrefix] Name prefix of the directory that measurement processes
      can create and use to store temporary data.
      \item[referenceOutputFile] Reference output to validate the result of transformed
        programs.
      \item[numExecutionTimeMeasurements] Number of execution time measurements (the minimal measured time
        is taken as the fitness of a schedule.)
      \item[numCompilatonDurationMeasurements] Number of compilation duration measurements.
      \item[validateOutput] Validate computation results.
      \item[tilingPermitInnerSeq] Tell \textsc{Polyite} that hthe code optimizer will tile inner fused loops.
      \item[measureParExecTime] enables parallel execution time measurement (true/ false)
      \item[measureSeqExecTime] enables sequential execution time measurement (true/ false)
      \item[seqPollyOptFlags] Polly flags for sequential code generation
      \item[parPollyOptFlags] Polly flags for parallel code generation
      \item[measureCacheHitRatePar] Measure the cache hit rate (using PAPI) of parallel execution
      \item[measureCacheHitRateSeq] Measure the cache hit rate (using PAPI) of sequential execution
      \item[benchmarkingSurrenderTimeout] If \textsc{Polyite} fails to start a benchmarking
      process, it will retry every 15 minutes but fails after the given timeout (in seconds). May be NONE.
    \end{description}

    \section{Data Export /Import}
    \begin{description}
      \item[populationFilePrefix] Name prefix for the population JSON file
      \item[importScheds] Import schedules from existing JSON file (the name is
      deduced from populationFilePrefix) (true/ false)
      \item[filterImportedPopulation] Do not reevaluate imported schedules where
      the evaluation failed previously.

      \item[exportSchedulesToJSCOPFiles] Export each schedule to a separate extended JSCOP file. (true/ false)
      \item[jscopFolderPrefix] Name prefix for the folders containing the JSCOP files. The name of files is generate
      from the imported JSCOP file's name.
      \item[exportPopulationToCSV] Export generated schedules and evaluation results to CSV
      \item[csvFilePrefix] Path prefix for the CSV files.
    \end{description}
\end{document}
